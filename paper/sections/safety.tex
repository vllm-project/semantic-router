% sections/safety.tex

\section{Request-Time Safety: Jailbreak and PII Signals}
\label{sec:safety}

Request-time safety in our architecture is implemented as \emph{first-class signals}, not as serial plugins.
Jailbreak and PII detection run in the signal extraction layer (\Cref{sec:signal_engine}), evaluated in parallel alongside all other signal types (keyword, embedding, domain, etc.).
When a safety signal fires, the decision engine matches it to a decision that activates the \texttt{fast\_response} plugin (\Cref{sec:plugins}), returning an immediate OpenAI-compatible response without contacting any upstream LLM.

This signal-driven design has three key advantages:
\begin{enumerate}[nosep,leftmargin=1.5em]
  \item \textbf{Zero added latency}: safety classifiers run concurrently with intent signals, so the wall-clock cost is max(safety, intent) rather than safety + intent.
  \item \textbf{Composability}: safety signals can be combined with domain, keyword, or embedding signals in decision rules using AND/OR logic---e.g., strict jailbreak detection only for financial queries.
  \item \textbf{Uniform observability}: safety results appear as standard signal matches in headers (\texttt{x-vsr-matched-jailbreak}, \texttt{x-vsr-matched-pii}), traces, and the dashboard topology view.
\end{enumerate}

\subsection{Jailbreak Detection}

Jailbreak attacks attempt to override model safety instructions through adversarial prompt construction.
Our detection pipeline addresses this as a classification problem with per-rule sensitivity control.

\textbf{Formulation.}
Given a text input $x$ (the user's latest message, or the full conversation history when \texttt{include\_history} is enabled), a classifier $g_\text{jb}$ produces:
\begin{equation}
  g_\text{jb}(x) = (y, c) \in \{\textsc{benign}, \textsc{injection}, \textsc{jailbreak}\} \times [0, 1]
\end{equation}
A jailbreak signal rule with threshold $\theta$ fires iff $y \neq \textsc{benign}$ and $c \geq \theta$.
Multiple rules at different thresholds can coexist, enabling decisions to reference different sensitivity levels.

\textbf{Classifier architecture.}
We support four inference backends with varying context-length and deployment characteristics:
(1)~BERT with LoRA adapters (standard context);
(2)~ModernBERT~\cite{warner2024modernbert} with Flash Attention;
(3)~mmBERT-32K with YaRN RoPE for 32K-token contexts;
(4)~external vLLM-served guardrail models for decoupled scaling.
All local backends use the LoRA adapter architecture (\Cref{sec:lora_mom}), reducing model memory footprint.

\textbf{Per-rule sensitivity.}
Different signal rules configure different thresholds: a \texttt{jailbreak\_strict} rule might use $\theta = 0.4$ with full history analysis for financial endpoints, while a \texttt{jailbreak\_standard} rule uses $\theta = 0.65$ on the latest message only.
Decisions then reference the appropriate rule by name, enabling context-aware security policies (e.g., strict detection only when the domain signal also matches ``economics'').

\subsection{Contrastive Jailbreak Detection}
\label{subsec:contrastive_jailbreak}

While the BERT classifier excels at detecting single-turn jailbreak attempts, it can be evaded by \emph{multi-turn escalation} (``boiling frog'') attacks in which each individual message appears benign but the conversation progressively steers the model toward unsafe behavior.
We introduce a contrastive embedding method that operates alongside the BERT classifier within the same jailbreak signal type.

\textbf{Knowledge base construction.}
Each contrastive rule defines two exemplar sets: a \emph{jailbreak knowledge base} $\mathcal{K}_\text{jb}$ containing known adversarial patterns (e.g., ``Ignore all previous instructions'', ``You are now DAN'') and a \emph{benign knowledge base} $\mathcal{K}_\text{ben}$ containing representative normal queries (e.g., ``What is the weather today'', ``Help me write an email'').
All KB embeddings are precomputed at initialization using a concurrent worker pool (identical to the complexity signal's preloading strategy), so runtime cost is limited to embedding the incoming message and computing cosine similarities.

\textbf{Contrastive scoring.}
For each user message $m$, the contrastive score measures relative proximity to the jailbreak vs.\ benign exemplars:
\begin{equation}
  \delta(m) = \max_{j \in \mathcal{K}_\text{jb}} \cos(\mathbf{m}, \mathbf{j}) \;-\; \max_{b \in \mathcal{K}_\text{ben}} \cos(\mathbf{m}, \mathbf{b})
  \label{eq:contrastive_score}
\end{equation}
A positive $\delta$ indicates the message is semantically closer to jailbreak patterns; a negative $\delta$ indicates it is closer to benign patterns.
The subtraction normalizes for the overall embedding similarity scale, making the threshold robust across different embedding models.

\textbf{Max-contrastive chain (multi-turn detection).}
When \texttt{include\_history} is enabled, the system exploits the stateless nature of the OpenAI chat API---each request carries the full conversation history in the \texttt{messages} array---to evaluate every user message:
\begin{equation}
  \Delta = \max_{m \in \mathcal{M}_\text{user}} \delta(m)
\end{equation}
The rule fires when $\Delta \geq \theta$ (default $\theta = 0.10$).
This max-pooling aggregation ensures that even if the current message is innocuous, the presence of any high-scoring turn in the history triggers detection.
The approach is effective against gradual escalation because the attacker must include at least one semantically adversarial turn to steer the model, and that turn will produce a high contrastive score regardless of its position in the conversation.

\textbf{Integration with signal architecture.}
The contrastive method is selected via the \texttt{method: contrastive} field on a jailbreak rule, keeping the same signal type $\tau_\text{jb}$ and rule-name addressing used by the BERT classifier.
This means contrastive rules participate fully in the decision engine's Boolean logic.
A typical deployment combines both methods:
\begin{itemize}[nosep,leftmargin=1.5em]
  \item A BERT rule (\texttt{jailbreak\_standard}, $\theta = 0.65$) for fast, high-precision single-turn detection.
  \item A contrastive rule (\texttt{jailbreak\_multiturn}, $\theta = 0.10$, \texttt{include\_history: true}) for multi-turn chain detection.
  \item A decision with OR logic fires if \emph{either} method detects an attack, providing defense in depth.
\end{itemize}
The embedding model is inherited from the global \texttt{embedding\_models} configuration (consistent with the complexity signal), avoiding per-rule model specification.

\subsection{PII Detection}

PII detection identifies personally identifiable information at the token level and enforces configurable allow/deny policies.

\textbf{Formulation.}
A token classifier $g_\text{pii}$ operates on the input sequence $x = (x_1, \ldots, x_T)$ and produces BIO-tagged predictions:
\begin{equation}
  g_\text{pii}(x) = \bigl\{(i, j, \ell, c) \mid \text{span } x_i \ldots x_j \text{ is PII type } \ell \text{ with confidence } c\bigr\}
\end{equation}
where $\ell \in \{\textsc{person}, \textsc{email}, \textsc{phone}, \textsc{ssn}, \textsc{credit\_card}, \ldots\}$.

\textbf{Policy model.}
Each PII signal rule specifies a threshold $\theta$ and an optional allow-list $\mathcal{L}_\text{allow}$.
The rule fires when any detected entity $(\cdot, \cdot, \ell, c)$ satisfies $c \geq \theta$ and $\ell \notin \mathcal{L}_\text{allow}$.
This per-rule policy enables differentiated enforcement: a \texttt{pii\_deny\_all} rule blocks all entity types, while a \texttt{pii\_allow\_email} rule permits email addresses for appointment booking in medical contexts.

\subsection{Parallel Safety Evaluation}

Both jailbreak and PII classifiers launch as concurrent goroutines within the signal extraction layer, alongside all other signal evaluators.
The wall-clock time is dominated by the slowest signal evaluator rather than the sum of safety and intent classifiers.
Both use the LoRA adapter architecture (\Cref{sec:lora_mom}), and their results merge into the standard signal result $S(r)$ consumed by the decision engine.

When a safety signal fires, the matched decision activates its \texttt{fast\_response} plugin, which short-circuits the pipeline and returns an OpenAI-compatible 200 OK response (supporting both streaming SSE and non-streaming JSON) without forwarding to any upstream model.

\begin{figure}[t]
\centering
\begin{tikzpicture}[
    node distance=0.3cm,
    sig/.style={rectangle, draw, rounded corners=2pt,
                minimum height=0.45cm, minimum width=1.4cm,
                align=center, font=\scriptsize, inner sep=2pt},
    dec/.style={rectangle, draw, thick, rounded corners=3pt,
                minimum height=0.5cm, minimum width=2.0cm,
                align=center, font=\scriptsize, inner sep=3pt,
                fill=green!6},
    act/.style={rectangle, draw, rounded corners=2pt,
                minimum height=0.45cm, minimum width=1.8cm,
                align=center, font=\scriptsize, inner sep=2pt,
                fill=orange!8},
    arr/.style={->, >=stealth, thick},
    darr/.style={->, >=stealth, densely dashed, gray!50},
    grp/.style={rectangle, draw, dashed, rounded corners=4pt,
                inner sep=5pt, gray!40},
  ]

  % Request
  \node[sig, fill=black!4, minimum width=1.0cm] (req) at (0, 0) {Request $r$};

  % Signal layer — all parallel
  \node[sig, fill=blue!6] (kw) at (2.8, 1.5) {$\tau_{\text{kw}}$};
  \node[sig, fill=green!6] (dom) at (4.2, 1.5) {$\tau_{\text{dom}}$};
  \node[sig, fill=green!6] (emb) at (5.6, 1.5) {$\tau_{\text{emb}}$};
  \node[sig, fill=red!8] (jb) at (2.8, 0.7) {$\tau_{\text{jb}}$};
  \node[sig, fill=red!8] (pii) at (4.2, 0.7) {$\tau_{\text{pii}}$};
  \node[font=\scriptsize, text=gray] (dots) at (5.6, 0.7) {$\cdots$};

  % Parallel bounding box
  \begin{scope}[on background layer]
  \node[grp, fill=blue!2, fit=(kw)(dom)(emb)(jb)(pii)(dots),
        label={[font=\scriptsize\bfseries, text=blue!50, anchor=south]above:Parallel Signal Extraction}] (sbox) {};
  \end{scope}

  \draw[arr] (req.east) -- (sbox.west);

  % S(r)
  \node[sig, fill=black!4, minimum width=0.8cm] (sr) at (7.0, 1.1) {$S(r)$};
  \draw[arr] (sbox.east) -- (sr.west);

  % Decision layer
  \node[dec] (d1) at (3.2, -1.2) {\texttt{block\_jb\_finance}};
  \node[dec] (d2) at (6.0, -1.2) {\texttt{block\_jb}};
  \node[dec] (d3) at (9.0, -1.2) {\texttt{block\_pii\_health}};

  % Decision conditions
  \node[font=\tiny, text=gray, anchor=north] at (3.2, -1.55) {$\tau_{\text{jb}}$\,{\tiny(strict)} AND $\tau_{\text{dom}}$\,{\tiny(finance)}};
  \node[font=\tiny, text=gray, anchor=north] at (6.0, -1.55) {$\tau_{\text{jb}}$\,{\tiny(standard)}};
  \node[font=\tiny, text=gray, anchor=north] at (9.0, -1.55) {$\tau_{\text{pii}}$\,{\tiny(allow email)} AND $\tau_{\text{dom}}$\,{\tiny(health)}};

  \draw[arr] (sr.south) -- ++(0, -0.5) -| (d1.north);
  \draw[arr] (sr.south) -- ++(0, -0.5) -| (d2.north);
  \draw[arr] (sr.south) -- ++(0, -0.5) -| (d3.north);

  % fast_response actions
  \node[act] (fr1) at (3.2, -2.6) {fast\_response};
  \node[act] (fr2) at (6.0, -2.6) {fast\_response};
  \node[act] (fr3) at (9.0, -2.6) {fast\_response};

  \draw[arr] (d1.south) -- (fr1.north);
  \draw[arr] (d2.south) -- (fr2.north);
  \draw[arr] (d3.south) -- (fr3.north);

  % 200 OK
  \node[sig, fill=black!4, minimum width=1.2cm] (ok) at (6.0, -3.6) {200 OK};
  \draw[darr] (fr1.south) |- (ok.west);
  \draw[darr] (fr2.south) -- (ok.north);
  \draw[darr] (fr3.south) |- (ok.east);

  % Priority labels
  \node[font=\tiny, text=red!60, anchor=south] at (3.2, -0.85) {P:1001};
  \node[font=\tiny, text=red!60, anchor=south] at (6.0, -0.85) {P:1000};
  \node[font=\tiny, text=red!60, anchor=south] at (9.0, -0.85) {P:998};

\end{tikzpicture}
\caption{Signal-driven safety architecture.  Safety classifiers ($\tau_{\text{jb}}, \tau_{\text{pii}}$) run in the signal layer in parallel with intent signals, adding zero marginal latency.  Decision rules compose safety signals with domain signals using AND/OR logic, enabling context-aware policies (e.g., strict jailbreak detection only for finance, relaxed PII policy for healthcare).  High-priority safety decisions activate the \texttt{fast\_response} plugin to return an immediate 200 OK without contacting any upstream model.}
\label{fig:safety_architecture}
\end{figure}

\subsection{Composable Safety Design}

Placing safety detection in the signal layer enables a class of flexible security policies that are difficult to express in a serial plugin architecture.
We highlight three patterns enabled by this design:

\textbf{Context-aware thresholds.}
A single deployment can define multiple jailbreak rules at different sensitivity levels (e.g., $\theta = 0.4$ for strict, $\theta = 0.65$ for standard).
Decisions compose these with domain signals: a \texttt{block\_jb\_finance} decision uses the strict rule AND the ``economics'' domain signal, while a \texttt{block\_jb\_general} decision uses the standard rule alone.
This provides differentiated security posture without separate deployments.

\textbf{Per-domain PII policies.}
Different domains require different PII handling: a healthcare endpoint may allow email and phone number for appointment booking while blocking SSN, whereas a public chatbot blocks all PII types.
Each PII signal rule carries its own allow-list, and decisions combine the appropriate rule with the relevant domain signal.
This replaces a monolithic PII policy with a set of composable, domain-scoped rules.

\textbf{Graduated response.}
By defining multiple safety decisions at different priorities, the system can implement graduated responses: a high-confidence jailbreak ($c \geq 0.9$) triggers an immediate block (priority 1001), while a moderate-confidence detection ($0.4 \leq c < 0.9$) triggers a warning header without blocking (priority 500).
The \texttt{fast\_response} plugin handles blocking; a header-only decision allows the request to proceed with an observability annotation.
This graduated model reduces false-positive impact while maintaining security coverage.
